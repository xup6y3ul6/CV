%!TEX TS-program = xelatex
%!TEX encoding = UTF-8 Unicode
% Awesome CV LaTeX Template for CV/Resume
%
% This template has been downloaded from:
% https://github.com/posquit0/Awesome-CV
%
% Author:
% Claud D. Park <posquit0.bj@gmail.com>
% http://www.posquit0.com
%
%
% Adapted to be an Rmarkdown template by Mitchell O'Hara-Wild
% 23 November 2018
%
% Template license:
% CC BY-SA 4.0 (https://creativecommons.org/licenses/by-sa/4.0/)
%
%-------------------------------------------------------------------------------
% CONFIGURATIONS
%-------------------------------------------------------------------------------
% A4 paper size by default, use 'letterpaper' for US letter
\documentclass[11pt, a4paper]{awesome-cv}

% Configure page margins with geometry
\geometry{left=1.4cm, top=.8cm, right=1.4cm, bottom=1.8cm, footskip=.5cm}

% Specify the location of the included fonts
\fontdir[fonts/]

% Color for highlights
% Awesome Colors: awesome-emerald, awesome-skyblue, awesome-red, awesome-pink, awesome-orange
%                 awesome-nephritis, awesome-concrete, awesome-darknight

\definecolor{awesome}{HTML}{F47F24}

% Colors for text
% Uncomment if you would like to specify your own color
% \definecolor{darktext}{HTML}{414141}
% \definecolor{text}{HTML}{333333}
% \definecolor{graytext}{HTML}{5D5D5D}
% \definecolor{lighttext}{HTML}{999999}

% Set false if you don't want to highlight section with awesome color
\setbool{acvSectionColorHighlight}{true}

% If you would like to change the social information separator from a pipe (|) to something else
\renewcommand{\acvHeaderSocialSep}{\quad\textbar\quad}

\def\endfirstpage{\newpage}

%-------------------------------------------------------------------------------
%	PERSONAL INFORMATION
%	Comment any of the lines below if they are not required
%-------------------------------------------------------------------------------
% Available options: circle|rectangle,edge/noedge,left/right

\name{Tzu-Yao Lin}{}

\position{Master in Psychology, National Taiwan University}
\address{3F., No.~213-1, Wuling St., Anle Dist., Keelung City 204007,
Taiwan (R.O.C.)}

\mobile{+886 989 878 512}
\email{\href{mailto:r08227112@ntu.edu.tw}{\nolinkurl{r08227112@ntu.edu.tw}}}
\homepage{Cognitive Psychometrics Lab}
\github{xup6y3ul6}

% \gitlab{gitlab-id}
% \stackoverflow{SO-id}{SO-name}
% \skype{skype-id}
% \reddit{reddit-id}


\usepackage{booktabs}

\providecommand{\tightlist}{%
	\setlength{\itemsep}{0pt}\setlength{\parskip}{0pt}}

%------------------------------------------------------------------------------



% Pandoc CSL macros
\newlength{\cslhangindent}
\setlength{\cslhangindent}{1.5em}
\newlength{\csllabelwidth}
\setlength{\csllabelwidth}{3em}
\newenvironment{CSLReferences}[3] % #1 hanging-ident, #2 entry spacing
 {% don't indent paragraphs
  \setlength{\parindent}{0pt}
  % turn on hanging indent if param 1 is 1
  \ifodd #1 \everypar{\setlength{\hangindent}{\cslhangindent}}\ignorespaces\fi
  % set entry spacing
  \ifnum #2 > 0
  \setlength{\parskip}{#2\baselineskip}
  \fi
 }%
 {}
\usepackage{calc}
\newcommand{\CSLBlock}[1]{#1\hfill\break}
\newcommand{\CSLLeftMargin}[1]{\parbox[t]{\csllabelwidth}{#1}}
\newcommand{\CSLRightInline}[1]{\parbox[t]{\linewidth - \csllabelwidth}{#1}}
\newcommand{\CSLIndent}[1]{\hspace{\cslhangindent}#1}

\begin{document}

% Print the header with above personal informations
% Give optional argument to change alignment(C: center, L: left, R: right)
\makecvheader

% Print the footer with 3 arguments(<left>, <center>, <right>)
% Leave any of these blank if they are not needed
% 2019-02-14 Chris Umphlett - add flexibility to the document name in footer, rather than have it be static Curriculum Vitae
\makecvfooter
  {July 2021}
    {Tzu-Yao Lin}
  {\thepage}


%-------------------------------------------------------------------------------
%	CV/RESUME CONTENT
%	Each section is imported separately, open each file in turn to modify content
%------------------------------------------------------------------------------



\begin{spacing}{1.5} 

\hypertarget{statement-of-research-interests-preparation-and-goals}{%
\section{Statement of Research Interests, Preparation, and
Goals}\label{statement-of-research-interests-preparation-and-goals}}

\begingroup
\fontsize{14pt}{1em}\bodyfontlight\bfseries\color{text}
\hypertarget{research-interests}{%
\subsection{Research Interests}\label{research-interests}}
\endgroup

\begingroup
\fontsize{12pt}{1em}\bodyfontlight\mdseries\color{text}
I am intensely interested in the mathematical modeling of perception and
decision-making processes. I study the cultural consensus theory (CCT)
and Bayesian statistics recently. The CCT, developed by Batchelder and
colleagues in the mid‑1980s, is a cognitively driven methodology to
assess informants' consensus. In my research, I incorporate a
response-confidence embedded the (low-) threshold theory into the CCT
framework for assessing the consensus answers to items, item difficulty,
and informant knowledge. By using the hierarchical Bayesian structure, the
new model, extended from the general Condorcet model (Batchelder and Anders, 2012), 
can apply to the ordinal categorical data type and account for the 
heterogeneities between- and within-cultural groups.
In addition, I also studied decision-making in
behavioral economics, analysis of neuroimaging data, and the
psychophysical model of time perception. Professor Yung‑Fong Hsu's lab
in which I participate usually discusses choice behavior, mathematical
psychology, and Fechnarian psychophysics. I am always intrigued by those topics and willing to acquire new techniques. 
\endgroup

\begingroup
\fontsize{14pt}{1em}\bodyfontlight\bfseries\color{text}
\hypertarget{preparation}{%
\subsection{Preparation}\label{preparation}}
\endgroup

\begingroup
\fontsize{12pt}{1em}\bodyfontlight\mdseries\color{text}
I have been an active learner and outstanding in all coursework I
undertook. I took not only the quantitative psychology courses but also
advanced statistics courses at Department of Mathematics, such as
advanced statistical inference (I)/(II), regression analysis,
multivariate statistical analysis, and applied Bayesian statistical
analysis. I am also proficient in programming, like R and python, and
have four years of experience on research projects. Those experiences
equip me with sufficient quantitative ability and teamwork spirit to
attend this 2021 summer school. Before attending this summer school, I
will have finished reading selected references and other materials about 
modeling heterogeneity of behavior with order-constrained data. 
\endgroup


\begingroup
\fontsize{14pt}{1em}\bodyfontlight\bfseries\color{text}
\hypertarget{goals}{%
\subsection{Goals}\label{goals}}
\endgroup

\begingroup
\fontsize{12pt}{1em}\bodyfontlight\mdseries\color{text}
Learning more skills to clearly model human behavior and mental process, especially in decision-making, is my primary aim in this training
program. I thought that there are two alternative approaches for
modeling the binary choice: (1) accounting for the decision maker's
response bias with additional covariates; and (2) extending the model to
deal with individual preference and group consensus preference, like the
cultural consensus theory approach. Maybe we can develop an R package for 
QTEST in the feature.
\endgroup

\end{spacing}

\end{document}
